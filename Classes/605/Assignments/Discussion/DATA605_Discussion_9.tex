\documentclass[]{article}
\usepackage{lmodern}
\usepackage{amssymb,amsmath}
\usepackage{ifxetex,ifluatex}
\usepackage{fixltx2e} % provides \textsubscript
\ifnum 0\ifxetex 1\fi\ifluatex 1\fi=0 % if pdftex
  \usepackage[T1]{fontenc}
  \usepackage[utf8]{inputenc}
\else % if luatex or xelatex
  \ifxetex
    \usepackage{mathspec}
  \else
    \usepackage{fontspec}
  \fi
  \defaultfontfeatures{Ligatures=TeX,Scale=MatchLowercase}
\fi
% use upquote if available, for straight quotes in verbatim environments
\IfFileExists{upquote.sty}{\usepackage{upquote}}{}
% use microtype if available
\IfFileExists{microtype.sty}{%
\usepackage{microtype}
\UseMicrotypeSet[protrusion]{basicmath} % disable protrusion for tt fonts
}{}
\usepackage[margin=1in]{geometry}
\usepackage{hyperref}
\hypersetup{unicode=true,
            pdftitle={DATA605 Discussion 9},
            pdfauthor={Henry Otuadinma},
            pdfborder={0 0 0},
            breaklinks=true}
\urlstyle{same}  % don't use monospace font for urls
\usepackage{color}
\usepackage{fancyvrb}
\newcommand{\VerbBar}{|}
\newcommand{\VERB}{\Verb[commandchars=\\\{\}]}
\DefineVerbatimEnvironment{Highlighting}{Verbatim}{commandchars=\\\{\}}
% Add ',fontsize=\small' for more characters per line
\usepackage{framed}
\definecolor{shadecolor}{RGB}{248,248,248}
\newenvironment{Shaded}{\begin{snugshade}}{\end{snugshade}}
\newcommand{\AlertTok}[1]{\textcolor[rgb]{0.94,0.16,0.16}{#1}}
\newcommand{\AnnotationTok}[1]{\textcolor[rgb]{0.56,0.35,0.01}{\textbf{\textit{#1}}}}
\newcommand{\AttributeTok}[1]{\textcolor[rgb]{0.77,0.63,0.00}{#1}}
\newcommand{\BaseNTok}[1]{\textcolor[rgb]{0.00,0.00,0.81}{#1}}
\newcommand{\BuiltInTok}[1]{#1}
\newcommand{\CharTok}[1]{\textcolor[rgb]{0.31,0.60,0.02}{#1}}
\newcommand{\CommentTok}[1]{\textcolor[rgb]{0.56,0.35,0.01}{\textit{#1}}}
\newcommand{\CommentVarTok}[1]{\textcolor[rgb]{0.56,0.35,0.01}{\textbf{\textit{#1}}}}
\newcommand{\ConstantTok}[1]{\textcolor[rgb]{0.00,0.00,0.00}{#1}}
\newcommand{\ControlFlowTok}[1]{\textcolor[rgb]{0.13,0.29,0.53}{\textbf{#1}}}
\newcommand{\DataTypeTok}[1]{\textcolor[rgb]{0.13,0.29,0.53}{#1}}
\newcommand{\DecValTok}[1]{\textcolor[rgb]{0.00,0.00,0.81}{#1}}
\newcommand{\DocumentationTok}[1]{\textcolor[rgb]{0.56,0.35,0.01}{\textbf{\textit{#1}}}}
\newcommand{\ErrorTok}[1]{\textcolor[rgb]{0.64,0.00,0.00}{\textbf{#1}}}
\newcommand{\ExtensionTok}[1]{#1}
\newcommand{\FloatTok}[1]{\textcolor[rgb]{0.00,0.00,0.81}{#1}}
\newcommand{\FunctionTok}[1]{\textcolor[rgb]{0.00,0.00,0.00}{#1}}
\newcommand{\ImportTok}[1]{#1}
\newcommand{\InformationTok}[1]{\textcolor[rgb]{0.56,0.35,0.01}{\textbf{\textit{#1}}}}
\newcommand{\KeywordTok}[1]{\textcolor[rgb]{0.13,0.29,0.53}{\textbf{#1}}}
\newcommand{\NormalTok}[1]{#1}
\newcommand{\OperatorTok}[1]{\textcolor[rgb]{0.81,0.36,0.00}{\textbf{#1}}}
\newcommand{\OtherTok}[1]{\textcolor[rgb]{0.56,0.35,0.01}{#1}}
\newcommand{\PreprocessorTok}[1]{\textcolor[rgb]{0.56,0.35,0.01}{\textit{#1}}}
\newcommand{\RegionMarkerTok}[1]{#1}
\newcommand{\SpecialCharTok}[1]{\textcolor[rgb]{0.00,0.00,0.00}{#1}}
\newcommand{\SpecialStringTok}[1]{\textcolor[rgb]{0.31,0.60,0.02}{#1}}
\newcommand{\StringTok}[1]{\textcolor[rgb]{0.31,0.60,0.02}{#1}}
\newcommand{\VariableTok}[1]{\textcolor[rgb]{0.00,0.00,0.00}{#1}}
\newcommand{\VerbatimStringTok}[1]{\textcolor[rgb]{0.31,0.60,0.02}{#1}}
\newcommand{\WarningTok}[1]{\textcolor[rgb]{0.56,0.35,0.01}{\textbf{\textit{#1}}}}
\usepackage{graphicx,grffile}
\makeatletter
\def\maxwidth{\ifdim\Gin@nat@width>\linewidth\linewidth\else\Gin@nat@width\fi}
\def\maxheight{\ifdim\Gin@nat@height>\textheight\textheight\else\Gin@nat@height\fi}
\makeatother
% Scale images if necessary, so that they will not overflow the page
% margins by default, and it is still possible to overwrite the defaults
% using explicit options in \includegraphics[width, height, ...]{}
\setkeys{Gin}{width=\maxwidth,height=\maxheight,keepaspectratio}
\IfFileExists{parskip.sty}{%
\usepackage{parskip}
}{% else
\setlength{\parindent}{0pt}
\setlength{\parskip}{6pt plus 2pt minus 1pt}
}
\setlength{\emergencystretch}{3em}  % prevent overfull lines
\providecommand{\tightlist}{%
  \setlength{\itemsep}{0pt}\setlength{\parskip}{0pt}}
\setcounter{secnumdepth}{0}
% Redefines (sub)paragraphs to behave more like sections
\ifx\paragraph\undefined\else
\let\oldparagraph\paragraph
\renewcommand{\paragraph}[1]{\oldparagraph{#1}\mbox{}}
\fi
\ifx\subparagraph\undefined\else
\let\oldsubparagraph\subparagraph
\renewcommand{\subparagraph}[1]{\oldsubparagraph{#1}\mbox{}}
\fi

%%% Use protect on footnotes to avoid problems with footnotes in titles
\let\rmarkdownfootnote\footnote%
\def\footnote{\protect\rmarkdownfootnote}

%%% Change title format to be more compact
\usepackage{titling}

% Create subtitle command for use in maketitle
\providecommand{\subtitle}[1]{
  \posttitle{
    \begin{center}\large#1\end{center}
    }
}

\setlength{\droptitle}{-2em}

  \title{DATA605 Discussion 9}
    \pretitle{\vspace{\droptitle}\centering\huge}
  \posttitle{\par}
    \author{Henry Otuadinma}
    \preauthor{\centering\large\emph}
  \postauthor{\par}
      \predate{\centering\large\emph}
  \postdate{\par}
    \date{27/10/2019}


\begin{document}
\maketitle

\hypertarget{a-restaurant-feeds-400-customers-per-day.-on-the-average-20-percent-of-the-customers-order-apple-pie.}{%
\paragraph{14 A restaurant feeds 400 customers per day. On the average
20 percent of the customers order apple
pie.}\label{a-restaurant-feeds-400-customers-per-day.-on-the-average-20-percent-of-the-customers-order-apple-pie.}}

\#\#\#\#(a) Give a range (called a 95 percent confidence interval) for
the number of pieces of apple pie ordered on a given day such that you
can be 95 percent sure that the actual number will fall in this range.

\hypertarget{solution}{%
\subparagraph{Solution}\label{solution}}

\begin{Shaded}
\begin{Highlighting}[]
\NormalTok{customers=}\DecValTok{400}
\NormalTok{order_pie =}\StringTok{ }\FloatTok{0.2} 
\NormalTok{others =}\StringTok{ }\FloatTok{0.8}
\NormalTok{(mean_con <-}\StringTok{ }\NormalTok{customers}\OperatorTok{*}\NormalTok{order_pie)}
\end{Highlighting}
\end{Shaded}

\begin{verbatim}
## [1] 80
\end{verbatim}

\begin{Shaded}
\begin{Highlighting}[]
\NormalTok{(}\DataTypeTok{stdev =} \KeywordTok{sqrt}\NormalTok{((order_pie}\OperatorTok{*}\NormalTok{others)}\OperatorTok{/}\NormalTok{customers))}
\end{Highlighting}
\end{Shaded}

\begin{verbatim}
## [1] 0.02
\end{verbatim}

\#95\% confidence interval

\begin{Shaded}
\begin{Highlighting}[]
\NormalTok{(}\DataTypeTok{conf_interval =} \FloatTok{1.96} \OperatorTok{*}\StringTok{ }\NormalTok{stdev)}
\end{Highlighting}
\end{Shaded}

\begin{verbatim}
## [1] 0.0392
\end{verbatim}

Compute the Lower Limit

\begin{Shaded}
\begin{Highlighting}[]
\NormalTok{(}\DataTypeTok{lm =}\NormalTok{ order_pie }\OperatorTok{-}\StringTok{ }\NormalTok{conf_interval)}
\end{Highlighting}
\end{Shaded}

\begin{verbatim}
## [1] 0.1608
\end{verbatim}

Comput Upper Limit

\begin{Shaded}
\begin{Highlighting}[]
\NormalTok{(}\DataTypeTok{ul=}\NormalTok{ order_pie }\OperatorTok{+}\StringTok{ }\NormalTok{conf_interval)}
\end{Highlighting}
\end{Shaded}

\begin{verbatim}
## [1] 0.2392
\end{verbatim}

Lower limit on number of customers

\begin{Shaded}
\begin{Highlighting}[]
\NormalTok{(}\DataTypeTok{l =}\NormalTok{ customers }\OperatorTok{*}\StringTok{ }\NormalTok{lm)}
\end{Highlighting}
\end{Shaded}

\begin{verbatim}
## [1] 64.32
\end{verbatim}

Upper limit on number of customers

\begin{Shaded}
\begin{Highlighting}[]
\NormalTok{(}\DataTypeTok{u =}\NormalTok{ customers }\OperatorTok{*}\StringTok{ }\NormalTok{ul)}
\end{Highlighting}
\end{Shaded}

\begin{verbatim}
## [1] 95.68
\end{verbatim}

Therefore, the 95\% confidence interval is: (64.32 , 95.68 )

\hypertarget{b-how-many-customers-must-the-restaurant-have-on-the-average-to-be-at-least-95-percent-sure-that-the-number-of-customers-ordering-pie-on-that-day-falls-in-the-19-to-21-percent-range}{%
\paragraph{(b) How many customers must the restaurant have, on the
average, to be at least 95 percent sure that the number of customers
ordering pie on that day falls in the 19 to 21 percent
range?}\label{b-how-many-customers-must-the-restaurant-have-on-the-average-to-be-at-least-95-percent-sure-that-the-number-of-customers-ordering-pie-on-that-day-falls-in-the-19-to-21-percent-range}}

\hypertarget{solution-1}{%
\subparagraph{Solution}\label{solution-1}}

\begin{Shaded}
\begin{Highlighting}[]
\NormalTok{l1=.}\DecValTok{19}
\NormalTok{order_pie =}\StringTok{ }\FloatTok{0.2}
\NormalTok{others =}\StringTok{ }\FloatTok{0.8}
\NormalTok{stdev =}\StringTok{ }\NormalTok{(order_pie }\OperatorTok{-}\NormalTok{l1)}\OperatorTok{/}\DecValTok{2}

\NormalTok{customers =}\StringTok{ }\NormalTok{(order_pie}\OperatorTok{*}\NormalTok{others)}\OperatorTok{/}\StringTok{ }\NormalTok{stdev}\OperatorTok{^}\DecValTok{2}

\KeywordTok{paste0}\NormalTok{(}\StringTok{'The restaurant must have '}\NormalTok{, }\KeywordTok{round}\NormalTok{(customers, }\DecValTok{0}\NormalTok{), }\StringTok{' customers'}\NormalTok{)}
\end{Highlighting}
\end{Shaded}

\begin{verbatim}
## [1] "The restaurant must have 6400 customers"
\end{verbatim}


\end{document}
